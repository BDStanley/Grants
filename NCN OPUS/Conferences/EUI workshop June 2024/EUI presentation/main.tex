\documentclass{beamer}
\usepackage{booktabs}
\usetheme{Boadilla}
\usecolortheme{dolphin}
\usefonttheme{professionalfonts}
\setbeamertemplate{navigation symbols}{}
\usepackage[utf8]{inputenc}

\AtBeginSection[]{
  \begin{frame}
  \vfill
  \centering
  \begin{beamercolorbox}[sep=8pt,center,shadow=false,rounded=false]{title}
    \usebeamerfont{title}\insertsectionhead\par%
  \end{beamercolorbox}
  \vfill
  \end{frame}
}

\title[Liberal-democratic hypocrisy?]{Liberal-democratic hypocrisy? \\Stability and change in Polish citizens' attitudes towards liberal democracy}
\author{Ben Stanley}
\institute{SWPS University}
\titlegraphic{\includegraphics[scale=0.3]{SWPS.jpg}}
\date{27 June, 2024}

\begin{document}

\begin{frame}
\titlepage
\end{frame}

\begin{frame}
\frametitle{Democratic hypocrisy and post-illiberal reform}
\begin{itemize}
    \item Do Polish voters hold stable attitudes to liberal democracy regardless of who holds power? 
    \item Or are they ``democratic hypocrites'', condemning and punishing illiberal actions when their favoured parties are in opposition, but failing to do so -- or even rewarding these actions -- when their favoured party is in power?
\end{itemize}
\end{frame}

\begin{frame}
\frametitle{The post-illiberal trilemma}
\begin{itemize}
    \item Fundamental problem faced by the new Polish government
    \item It is expected to act
    \begin{itemize}
        \item (a) effectively, to deal with the consequences of illiberalism
        \item (b) quickly, to satisfy popular demands for change and justice
        \item (c) legally, to ensure that the rule of law is restored
    \end{itemize}   
    \item However, in many cases it can do only two of these things at once.
\end{itemize}
\end{frame}

\begin{frame}
\frametitle{The post-illiberal trilemma}
\begin{figure}
    \centering
    \includegraphics[width=0.8\linewidth]{Trilemma.png}
\end{figure}
\end{frame}

\begin{frame}
\frametitle{Vulnerability to ``eroding from the top''}
\begin{itemize}
    \item ``Pernicious polarization” (Somer et al., 2021) trumps respect for the system of rights and restraints through which that political system functions. 
    \item ``Authoritarian winners'' (Cohen et al. 2023) whose support for the political system is contingent.
    \item Greater propensity of opposition supporters to stand up for democratic checks and balances (Mazepus and Toshkov 2022).
    \item  Şaşmaz et al. (2022) and Fossati (2021) find greater acceptance among voters of anti-democratic actions if those actions benefit their preferred party.
    \item  In the presence of strong partisanship democratic principles may be sacrificed to partisan interests (Bryan 2023), even in unstable party systems with considerable voter volatility (Hrbková et al. 2023).
\end{itemize}
\end{frame}

\begin{frame}
\frametitle{Democratic hypocrisy}
\begin{itemize}
    \item Why do voters fail to punish ``erosion from the top''?
    \item Svolik (2019)
    \begin{itemize}
    \item voters ``have not had enough time or clarity to recognize a subversion of democracy for what it is'';
    \item voters ``do not care much about democracy in the first place'';
    \item ``deep social cleavages and acute political tensions ... undercut the public’s ability to curb the illiberal inclinations of elected politicians.''
    \end{itemize}
    \item Simonovits et al. (2022) -- voters may be “democratic hypocrites” whose attitudes to liberal democracy -- and the ways in which they act upon those attitudes -- are subordinated to party-political concerns.
\end{itemize}
\end{frame}

\begin{frame}
\frametitle{Hypotheses}
After a change of government...
\begin{itemize}
    \item ...PiS voters’ level of agreement with liberal-democratic principles will increase (\emph{H1a}), while KO, Lewica and Trzecia Droga voters’ level of agreement with liberal-democratic values will decrease (\emph{H1b});
    \item ...PiS’s voters will become more negative in their evaluations of the state of Polish democracy (\emph{H2a}), and KO-Lewica-TD voters will become more positive (\emph{H2b});
    \item ...PiS voters will be more likely to punish illiberal candidates in the third wave of the survey (\emph{H3a}), while KO, Lewica and TD voters will become less lenient (\emph{H3b}).
\end{itemize}
\end{frame}

\begin{frame}
\frametitle{Data and methods}
\begin{itemize}
    \item Three-wave CAWI panel survey (April 2023, October 2023, April 2024)
    \item H1 \& H2 - OLS / ordinal regression models, regressing liberal democratic attitudes (Claassen et al. 2024; ESS R10) on vote choice.
    \item H3 - forced-choice conjoint; impact of illiberal positions taken by candidates on the probability of voting for a candidate, conditional on party electorate.
\end{itemize}
\end{frame}

\begin{frame}
\frametitle{Support for liberal democracy}
\begin{figure}
    \centering
    \includegraphics[width=1\linewidth]{libdem_change_plot_present.png}
\end{figure}
\end{frame}

\begin{frame}
\frametitle{Agreement with liberal-democratic principles}
\begin{figure}
    \centering
    \includegraphics[width=1\linewidth]{claassen_most_lib_position_present.png}
\end{figure}
\end{frame}

\begin{frame}
\frametitle{Importance of liberal-democratic principles}
\begin{figure}
    \centering
    \includegraphics[width=1\linewidth]{essgen_most_lib_position_present.png}
\end{figure}
\end{frame}

\begin{frame}
\frametitle{Perception of realisation of liberal-democratic principles}
\begin{figure}
    \centering
    \includegraphics[width=1\linewidth]{esspol_most_lib_position_present.png}
\end{figure}
\end{frame}

\begin{frame}
\frametitle{Probability of voting for illiberal candidates}
\begin{figure}
    \centering
    \includegraphics[width=1\linewidth]{ko_conjoint.png}
\end{figure}
\end{frame}

\begin{frame}
\frametitle{Probability of voting for illiberal candidates}
\begin{figure}
    \centering
    \includegraphics[width=1\linewidth]{lewica_conjoint.png}
\end{figure}
\end{frame}

\begin{frame}
\frametitle{Probability of voting for illiberal candidates}
\begin{figure}
    \centering
    \includegraphics[width=1\linewidth]{pis_conjoint.png}
\end{figure}
\end{frame}

\begin{frame}
\frametitle{Conclusions}
\begin{itemize}
    \item Polish democracy may have been eroded from the top, but this erosion was facilitated by weaknesses in the structure of liberal-democratic values. 
    \item A majority of Poles remain nominally pro-democratic.
    \item However, their attitudes are strongly influenced by party affiliation and political context. 
    \item A substantial proportion of the electorate holds positions on liberal democracy that smack more of pragmatism than principle.
    \item This raises the possibility that Polish democracy may settle into a post-illiberal equilibrium, with pro-government partisans content with a semi-reformed system in which efficient executive decisionism is relatively unencumbered by judicial control.
\end{itemize}
\end{frame}

\end{document}
